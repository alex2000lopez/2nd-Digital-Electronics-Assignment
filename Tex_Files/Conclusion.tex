\section{Conclusion}

In a nutshell, this project has posed an interesting challenge and it has given us the oportunity to not only learn about VHDL in the process but also to work with the different modules and ICs that we have seen in the course.\medskip

Out system is able to perform the duties that we initially intended it to perform even after taking into account the limitations of the hardware that we have used. The restrictions of the GAL PLDs in terms of input/output ports, memory capacity and the programming language, which was new to us in a sense, have made us use a considerable number of GALs, which only proves how far technology has come in the past few years. This project could have been implemented more easily using CPLDs or even FPGAs or, in the other side of the spectrum, a simple microcontroller. \medskip

Even though we have managed to create a fully functional safe deposit box, there are lots of things to improve nonetheless:

\begin{itemize}
    \item Passwords can only have 4 digits, and the same number cannot be repeated consecutively.
    
    \item To introduce and validate a password one must press \textit{*} and \textit{\#} , which makes the process slower.
    
    \item The code is most probably not as efficient as it could be. Improving it would eliminate some PLDs making the device not only more compact but also cheaper and easier to implement.
    
    \item The set password mechanism requires the user to know the binary code of the password, which is something a bit inconvenient.
    
    \item As we have said before, the use of CPLDs, FPGAs, or microcontrollers would greatly help reduce the size of the project.
\end{itemize}


All in all, we think that this project has served not only to wrap up the knowledge that we have acquired in this subject but also to learn to work as a team, mostly due to the complexity of certain parts. We sincerely hope you enjoy it as much as we have enjoyed putting it together. 